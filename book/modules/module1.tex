\Heading{Modeling}

Suppose you are observing some \emph{green} ants walking on the sidewalk.
In the first minute you record 10 ants. In the second minute you
record 20. In the third minute, you record 40 ants. This continues until
there are too many ants for you to count.

\begin{center}
	\begin{tabular}{c|c}
		Minute & \#Green Ants\\
		\hline
		1 & 10\\
		2 & 20\\
		3 & 40\\
		4 & 80\\
		\vdots & \vdots
	\end{tabular}
\end{center}

Since you lost count of the ants, you decide to use mathematics to try and figure out
how many ants walked by on minutes $5$, $6$, \ldots. You notice the pattern that
\[
	\text{Green ants per minute $n$} = 2^{n-1}\cdot 10.
\]
Stupendous! Mathematics now predicts there were $160$ ants during minute $5$. But something
else catches your eye. Across the sidewalk are \emph{brown} ants. You count these
ants every minute.

\begin{center}
	\begin{tabular}{c|c}
		Minute & \#Brown Ants\\
		\hline
		1 & 3\\
		2 & 6\\
		3 & 12\\
		4 & 24\\
		\vdots & \vdots
	\end{tabular}
\end{center}

The pattern is slightly different. This time, 
\[
	\text{Green ants per minute $n$} = 2^{n-1}\cdot 3.
\]

Your friend, who was watching you the whole time, looks confused. ``Why come up with two complicated equations
when you can describe both types of ant at once?'' they declare.

\begin{center}
	\begin{tabular}{c}
		$\text{\#Ants at minute $n$}\ =\ 2\cdot(\text{\#Ants at minute $n-1$})$\\
		$\text{\#Green ants at minute 1}=10$\\
		$\text{\#Brown ants at minute 1}=3$\\
	\end{tabular}
\end{center}

Your friend has a point. Their model is elegant, but \emph{your} model can predict how many ants pass by at minute $3.222$!
Though, your friend would probably complain that $46.654$ is not a number of ants\ldots.

\medskip

You and your friend have just come up with two different \emph{mathematical models} for the number of ants
that walk across the sidewalk. They happen to make similar predictions for each minute and each have their
strengths and weakenesses. In this course, we will be focused on a particular type of mathematical model---one
that uses \emph{differential equations} at its core.

\Heading{Types of Models}

\begin{definition}[Mathematical Model]
A \emph{mathematical model} is a description of the world
	\begin{enumerate}
		\item created in the service of answering a question, and
		\item where the complexity of the world has been abstracted away to numbers, quantities, and their relationships\footnote{Other
mathematical objects are also allowed.}.
	\end{enumerate}
\end{definition}

In the previous situation, the \emph{question} you were trying to answer was ``how many ants are there at a given minute?''.
And, we sidestepped difficult issues like, ``Is an ant that is missing three legs still an ant?'' by using the common-sense
convention that ``the number of ants is a whole number and one colored blob that moves under its own power corresponds to one ant''.





