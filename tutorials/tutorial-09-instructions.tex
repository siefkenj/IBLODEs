\subsection*{Learning Objectives}
	Students need to be able to\ldots
	\begin{itemize}
		\item Build a spreadsheet that can use Euler's method to approximate the solution to a system of differential equations
		\item Rewrite a higher-order ODE as a system of ODEs.
		\item Use Euler's method to simulate the solution to a higher-order ODE.
		\item Numerically approximate the solution to a boundary value problem involving higher-order ODEs.
	\end{itemize}

\subsection*{Context}
	
In class we used Euler's method extensively.

We also introduced boundary-value problems (BVPs) in class to show that studying existence and uniqueness of solution is not a trivial matter.

For this type of differential equations, however, we can't use Euler's method directly, since it requires initial conditions.
This type of ODEs also requires a higher-order ODE.

In this tutorial, we will do the following:
\begin{itemize}
	\item Write a second-order ODE as a system of first-order ODEs
	\item Simulate a second-order ODE (as a system of ODEs) using Euler's method with initial conditions in Excel
	\item Develop a method to approximate solutions to a BVP
\end{itemize}


\subsection*{Resources for TAs}

Excel spreadsheet with the Shooting Method. 

\begin{itemize}
	\item \url{https://utoronto-my.sharepoint.com/:x:/g/personal/bernardo_galvao_sousa_utoronto_ca/EQO08o4-PdFMr8PMd80LmP0BPilrLuGIcNMpP-oVpfM4ow?e=IR39N4}
\end{itemize}


\subsection*{Before Tutorial}

Send an announcement to students letting them know that they will need to bring a laptop and will be using Excel and Euler's method in tutorial.


\subsection*{What to Do}
	Introduce the learning objectives for the day's tutorial. \\
	
	Explain that we will start by simulating Euler's method for a second-order ODE by writing it as a first-order system.
	
	Have students get into small groups and start on \#1 -- each group needs to have at least 1 laptop. 
	
	This question should be done quickly - it's setting the stage for \#2. \\
	
	
	Explain the idea of the shooting method:
	\begin{enumerate}
		\item Start with two guesses, each on a different side of the target
		\item Next guess is the average of the previous two $\to$ check how close/far it gets from the target
		\item From the previous three guesses, choose the two that have the target in between
		\item Go back to Step 2
	\end{enumerate}
	
	This question has many parts and should take most students the whole tutorial. \\

	The last question is a challenge for quicker students or for students to try at home.
	


%\subsection*{Notes}
%
%	\begin{enumerate}
%		\item Note
%	\end{enumerate}
	
	

	
	
